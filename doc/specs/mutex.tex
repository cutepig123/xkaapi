%%%%%%%%%%%%%%%%%%%%%%%%%%%%%%%%%%%%%%%%%%%%%%%%%%%%%%%%%%%%%%%%%%%%%%%%%%%%%%
% MUTEX
%%%%%%%%%%%%%%%%%%%%%%%%%%%%%%%%%%%%%%%%%%%%%%%%%%%%%%%%%%%%%%%%%%%%%%%%%%%%%%
Our mutexes are implemented same way as POSIX mutexes are blablabla. We do not
currently implement the following  function :

\begin{verbatim}
int   kaapi_mutex_consistent(kaapi_mutex_t *);

int   kaapi_mutex_getprioceiling(const kaapi_mutex_t *restrict,
          int *restrict);

int   kaapi_mutex_setprioceiling(kaapi_mutex_t *restrict, int,
          int *restrict);

int   kaapi_mutex_timedlock(kaapi_mutex_t *restrict,
          const struct timespec *restrict);

int   kaapi_mutexattr_getprioceiling(
          const kaapi_mutexattr_t *restrict, int *restrict);

int   kaapi_mutexattr_getprotocol(const kaapi_mutexattr_t *restrict,
          int *restrict);

int   kaapi_mutexattr_getpshared(const kaapi_mutexattr_t *restrict,
          int *restrict);

int   kaapi_mutexattr_getrobust(const kaapi_mutexattr_t *restrict,
          int *restrict);

int   kaapi_mutexattr_setprioceiling(kaapi_mutexattr_t *, int);

int   kaapi_mutexattr_setprotocol(kaapi_mutexattr_t *, int);

int   kaapi_mutexattr_setpshared(kaapi_mutexattr_t *, int);

int   kaapi_mutexattr_setrobust(kaapi_mutexattr_t *, int);
\end{verbatim}

\begin{description}
\vspace*{3ex} \hrule width 4cm
\vspace*{3ex}
\item [\texttt{int kaapi\_mutexattr\_init (kaapi\_mutexattr\_t *attr);}]
\item [\texttt{int kaapi\_mutexattr\_destroy (kaapi\_mutexattr\_t *attr);}]~\\
%\textbf{\texttt{int kaapi\_mutexattr\_destroy (kaapi\_mutexattr\_t *attr)}} :

\paragraph{Description}~\\
The kaapi\_mutexattr\_destroy() function shall destroy a mutex attributes
object. A destroyed attr attributes object can be reinitialized using
kaapi\_mutexattr\_init(); the results of otherwise referencing the object
after it has been destroyed are undefined.

The kaapi\_mutexattr\_init() function shall initialize a mutex attributes
object attr with the default value for all of the attributes defined by the
implementation.

Results are undefined if kaapi\_mutexattr\_init() is called specifying an
already initialized attr attributes object.

After a mutex attributes object has been used to initialize one or more
mutexes, any function affecting the attributes object (including destruction)
shall not affect any previously initialized mutexes.

\paragraph{Return Value}~\\
Upon successful completion, kaapi\_mutexattr\_destroy() and
kaapi\_mutexattr\_init() shall return zero; otherwise, an error number shall
be returned to indicate the error.

\paragraph{Errors}~\\
The kaapi\_mutexattr\_destroy() function may fail if:

\begin{description}
\item [KAAPI\_EINVAL]: The value specified by attr is invalid.
\end{description}

The kaapi\_mutexattr\_init() function shall not fail.

These functions shall not return an error code of \verb+KAAPI_ENOMEM+.
\end{description}

%%%%%%%%%%%%%%%%%%%%%%%%%%%%%%%%%%%%%%%%%%%%%%%%%%%%%%%%%%%%%%%%%%%%%%%%%%%%%%

\begin{description}
\vspace*{3ex} \hrule width 4cm
\vspace*{3ex} 
\item [\texttt{int kaapi\_mutexattr\_gettype (const kaapi\_mutexattr\_t
    *\_\_restrict, int *\_\_restrict);}]
\item [\texttt{int kaapi\_mutexattr\_settype (kaapi\_mutexattr\_t *, int);}]~\\

\paragraph{Description}~\\
The kaapi\_mutexattr\_gettype() and kaapi\_mutexattr\_settype() functions,
respectively, shall get and set the mutex type attribute. This attribute is
set in the type parameter to these functions. The default value of the type
attribute is KAAPI\_MUTEX\_NORMAL.

The type of mutex is contained in the type attribute of the mutex attributes.

The type of mutex is contained in the type attribute of the mutex
attributes. Valid mutex types include:

KAAPI\_MUTEX\_NORMAL: This type of mutex does not detect deadlock. A thread
attempting to relock this mutex without first unlocking it shall
deadlock. Attempting to unlock a mutex locked by a different thread results in
undefined behavior. Attempting to unlock an unlocked mutex results in
undefined behavior.

KAAPI\_MUTEX\_RECURSIVE: A thread attempting to relock this mutex without
first unlocking it shall succeed in locking the mutex. Multiple locks of this
mutex shall require the same number of unlocks to release the mutex before
another thread can acquire the mutex. A thread attempting to unlock a mutex
which another thread has locked shall return with an error. A thread
attempting to unlock an unlocked mutex shall return with an error.

\paragraph{Return Value}~\\
Upon successful completion, the kaapi\_mutexattr\_gettype() function shall
return zero and store the value of the type attribute of attr into the object
referenced by the type parameter. Otherwise, an error shall be returned to
indicate the error.

If successful, the kaapi\_mutexattr\_settype() function shall return zero;
otherwise, an error number shall be returned to indicate the error.

\paragraph{Errors}~\\
The kaapi\_mutexattr\_settype() function shall fail if:

\begin{description}
\item [KAAPI\_EINVAL]: The value type is invalid.
\end{description}

The kaapi\_mutexattr\_gettype() and kaapi\_mutexattr\_settype() functions may
fail if:

\begin{description}
\item [KAAPI\_EINVAL]: The value specified by attr is invalid.
\end{description}
\end{description}

%% Mutex Attributes Not Implemented here?

%%%%%%%%%%%%%%%%%%%%%%%%%%%%%%%%%%%%%%%%%%%%%%%%%%%%%%%%%%%%%%%%%%%%%%%%%%%%%%
% MUTEX MANAGEMENT
%%%%%%%%%%%%%%%%%%%%%%%%%%%%%%%%%%%%%%%%%%%%%%%%%%%%%%%%%%%%%%%%%%%%%%%%%%%%%%

\begin{description}
\vspace*{3ex} \hrule width 4cm
\vspace*{3ex} 
\item [\texttt{int kaapi\_mutex\_init (kaapi\_mutex\_t *\_\_restrict mutex,}]~\\
\textbf{\texttt{ const kaapi\_mutexattr\_t *\_\_restrict attr);}}
\item [\texttt{int kaapi\_mutex\_destroy (kaapi\_mutex\_t *mutex);}]
\item [\texttt{kaapi\_mutex\_t mutex = KAAPI\_MUTEX\_INITIALIZER;}]~\\

%\textbf{\texttt{int kaapi\_mutexattr\_destroy (kaapi\_mutexattr\_t *attr)}} :

\paragraph{Description}~\\
The kaapi\_mutex\_destroy() function shall destroy the mutex object referenced
by mutex; the mutex object becomes, in effect, uninitialized. An
implementation may cause kaapi\_mutex\_destroy() to set the object referenced
by mutex to an invalid value. A destroyed mutex object can be reinitialized
using kaapi\_mutex\_init(); the results of otherwise referencing the object
after it has been destroyed are undefined.

It shall be safe to destroy an initialized mutex that is unlocked. Attempting
to destroy a locked mutex results in undefined behavior.

The kaapi\_mutex\_init() function shall initialize the mutex referenced by
mutex with attributes specified by attr. If attr is NULL, the default mutex
attributes are used; the effect shall be the same as passing the address of a
default mutex attributes object. Upon successful initialization, the state of
the mutex becomes initialized and unlocked.

Only mutex itself may be used for performing synchronization. The result of
referring to copies of mutex in calls to kaapi\_mutex\_lock(),
kaapi\_mutex\_trylock(), kaapi\_mutex\_unlock(), and kaapi\_mutex\_destroy()
is undefined.

Attempting to initialize an already initialized mutex results in undefined
behavior.

In cases where default mutex attributes are appropriate, the macro
KAAPI\_MUTEX\_INITIALIZER can be used to initialize mutexes that are
statically allocated. The effect shall be equivalent to dynamic initialization
by a call to kaapi\_mutex\_init() with parameter attr specified as NULL,
except that no error checks are performed.

\paragraph{Return Value}~\\
If successful, the kaapi\_mutex\_destroy() and kaapi\_mutex\_init() functions
shall return zero; otherwise, an error number shall be returned to indicate
the error.

\paragraph{Errors}~\\
The kaapi\_mutex\_destroy() function may fail if:

\begin{description}
\item [KAAPI\_EBUSY]: The implementation has detected an attempt to destroy
  the object referenced by mutex while it is locked or referenced (for
  example, while being used in a kaapi\_cond\_timedwait() or
  kaapi\_cond\_wait()) by another thread.
\item [KAAPI\_EINVAL]: The value specified by mutex is invalid.
\end{description}

The kaapi\_mutex\_init() function shall fail if:

\begin{description}
\item [KAAPI\_EAGAIN]: The system lacked the necessary resources (other than
  memory) to initialize another mutex.
\item [KAAPI\_ENOMEM]: Insufficient memory exists to initialize the mutex.
\item [KAAPI\_EPERM]: The caller does not have the privilege to perform the
  operation.
\end{description}

The kaapi\_mutex\_init() function may fail if:

\begin{description}
\item [KAAPI\_EBUSY]: The implementation has detected an attempt to
  reinitialize the object referenced by mutex, a previously initialized, but
  not yet destroyed, mutex.
\item [KAAPI\_EINVAL]: The value specified by attr is invalid.
\end{description}
\end{description}

%%%%%%%%%%%%%%%%%%%%%%%%%%%%%%%%%%%%%%%%%%%%%%%%%%%%%%%%%%%%%%%%%%%%%%%%%%%%%%

\begin{description}
\vspace*{3ex} \hrule width 4cm
\vspace*{3ex} 
\item [\texttt{int kaapi\_mutex\_lock (kaapi\_mutex\_t *mutex);}]
\item [\texttt{int kaapi\_mutex\_trylock (kaapi\_mutex\_t *mutex);}]
\item [\texttt{int kaapi\_mutex\_unlock (kaapi\_mutex\_t *mutex);}]~\\
%\textbf{\texttt{int kaapi\_mutexattr\_destroy (kaapi\_mutexattr\_t *attr)}} :

\paragraph{Description}~\\
The mutex object referenced by mutex shall be locked by calling
kaapi\_mutex\_lock(). If the mutex is already locked, the calling thread shall
block until the mutex becomes available. This operation shall return with the
mutex object referenced by mutex in the locked state with the calling thread
as its owner.

[XSI] [Option Start] If the mutex type is KAAPI\_MUTEX\_NORMAL, deadlock
detection shall not be provided. Attempting to relock the mutex causes
deadlock. If a thread attempts to unlock a mutex that it has not locked or a
mutex which is unlocked, undefined behavior results.

If the mutex type is KAAPI\_MUTEX\_ERRORCHECK, then error checking shall be
provided. If a thread attempts to relock a mutex that it has already locked,
an error shall be returned. If a thread attempts to unlock a mutex that it has
not locked or a mutex which is unlocked, an error shall be returned.

If the mutex type is KAAPI\_MUTEX\_RECURSIVE, then the mutex shall maintain
the concept of a lock count. When a thread successfully acquires a mutex for
the first time, the lock count shall be set to one. Every time a thread
relocks this mutex, the lock count shall be incremented by one. Each time the
thread unlocks the mutex, the lock count shall be decremented by one. When the
lock count reaches zero, the mutex shall become available for other threads to
acquire. If a thread attempts to unlock a mutex that it has not locked or a
mutex which is unlocked, an error shall be returned.

If the mutex type is KAAPI\_MUTEX\_DEFAULT, attempting to recursively lock the
mutex results in undefined behavior. Attempting to unlock the mutex if it was
not locked by the calling thread results in undefined behavior. Attempting to
unlock the mutex if it is not locked results in undefined behavior. [Option
  End]

The kaapi\_mutex\_trylock() function shall be equivalent to
kaapi\_mutex\_lock(), except that if the mutex object referenced by mutex is
currently locked (by any thread, including the current thread), the call shall
return immediately. If the mutex type is KAAPI\_MUTEX\_RECURSIVE and the mutex
is currently owned by the calling thread, the mutex lock count shall be
incremented by one and the kaapi\_mutex\_trylock() function shall immediately
return success.

The kaapi\_mutex\_unlock() function shall release the mutex object referenced
by mutex. [XSI] [Option Start] The manner in which a mutex is released is
dependent upon the mutex's type attribute. [Option End] If there are threads
blocked on the mutex object referenced by mutex when kaapi\_mutex\_unlock() is
called, resulting in the mutex becoming available, the scheduling policy shall
determine which thread shall acquire the mutex.

[XSI] [Option Start] (In the case of KAAPI\_MUTEX\_RECURSIVE mutexes, the
mutex shall become available when the count reaches zero and the calling
thread no longer has any locks on this mutex.) [Option End]

If a signal is delivered to a thread waiting for a mutex, upon return from the
signal handler the thread shall resume waiting for the mutex as if it was not
interrupted.

\paragraph{Return Value}~\\
If successful, the kaapi\_mutex\_lock() and kaapi\_mutex\_unlock() functions
shall return zero; otherwise, an error number shall be returned to indicate
the error.

The kaapi\_mutex\_trylock() function shall return zero if a lock on the mutex
object referenced by mutex is acquired. Otherwise, an error number is returned
to indicate the error.

\paragraph{Errors}~\\
The kaapi\_mutex\_lock() and kaapi\_mutex\_trylock() functions shall fail if:

\begin{description}
\item [KAAPI\_EINVAL]: The mutex was created with the protocol attribute
  having the value KAAPI\_PRIO\_PROTECT and the calling thread's priority is
  higher than the mutex's current priority ceiling.
\end{description}

The kaapi\_mutex\_trylock() function shall fail if:

\begin{description}
\item [KAAPI\_EBUSY]: The mutex could not be acquired because it was already
  locked.
\end{description}

The kaapi\_mutex\_lock(), kaapi\_mutex\_trylock(), and kaapi\_mutex\_unlock()
functions may fail if:

\begin{description}
\item [KAAPI\_EINVAL]: The value specified by mutex does not refer to an
  initialized mutex object.
\item [KAAPI\_EAGAIN]: [XSI] [Option Start] The mutex could not be acquired
  because the maximum number of recursive locks for mutex has been
  exceeded. [Option End]
\end{description}

The kaapi\_mutex\_lock() function may fail if:

\begin{description}
\item [KAAPI\_EDEADLK]: A deadlock condition was detected or the current
  thread already owns the mutex.
\end{description}

The kaapi\_mutex\_unlock() function may fail if:

\begin{description}
\item [KAAPI\_EPERM]
    The current thread does not own the mutex. 
\end{description}
\end{description}
