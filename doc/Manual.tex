\documentclass{article}
\usepackage[utf8x]{inputenc}

\usepackage{url} \urlstyle{sf}
\usepackage[a4paper,margin=1.9cm]{geometry}
\usepackage{xspace}
\usepackage[francais,american]{babel}
\usepackage{palatino}
\usepackage{bibtopic}
%\usepackage{graphicx}
%\usepackage{enumitem}
%\usepackage{dot2texi}

\newcommand{\kaapi}{\textsc{X}-Kaapi\xspace}
%%%\newcommand{\new}{\hspace*{10ex}\textbf{\textsc{New in \kaapi.}~\\}\xspace}
%\newcommand{\new}{}
%\newcommand{\inote}[1]{\textit{\textbf{  \center \hrule Implementation note\hrule}}\vspace*{1ex}\textit{#1}\vspace{1ex} \hrule
%\vspace*{2ex}}

%%\newcounter{subsubsection}[subsection]
%\renewcommand{\subsubsection}[1]{~\\ \addtocounter{subsubsection}{1} \noindent\textit{
%%\textbf{\thesubsection.
%\thesubsubsection. #1\\}}

%
%\newcounter{subsubsubsection}[subsubsection]
%\newcommand{\subsubsubsection}[1]{~\\ \addtocounter{subsubsubsection}{1} \noindent\textit{\textbf{\thesubsubsection.
%\thesubsubsubsection. #1\\}}}
%\newtheorem{proposition}{Proposition}

%%\renewcommand{\subsubsection}[1]{~\\ \addtocounter{subsubsection}{1} \noindent\textit{\textbf{\thesubsubsection\hspec #1\\}}}

\begin{document}

\title{X-Kaapi's User Manual}
\author{C. Laferrière}
\date{\today}
\maketitle
\newpage
\tableofcontents
\newpage
\section{Introduction}


Here is a short manual of the \kaapi library. This Manual will guide you through the installation of the library and its usage.
You may find contact information about people involved in this project in the \verb+AUTHORS+ file in the source directory.

Blabla sur le délivrable ...

\section{Installation}

In order to compile and use \kaapi's library, simply follow the next steps.

\subsection{Prerequisite}

\kaapi 's build system uses CMake \footnote{http://www.cmake.org/}. The minimum version of CMake you need is 2.6.

\subsection{Compilation}

You may as well compile the library inside or outside the source directory. In order to create Makefiles, you need to run the \verb+cmake+ command.
\\

For a compilation inside source directory :
\begin{verbatim}
$> cmake . -DKAAPI_TARGET_MT=1
\end{verbatim}


For a compilation outside source directory, for example you may have a "build" directory at the same level as the "libxkaapi-1.0.0-Source" directory :
\begin{verbatim}
$> cd build/
$> cmake ../libxkaapi-1.0.0-Source/ -DKAAPI_TARGET_MT=1
\end{verbatim}

The \verb+-DKAAPI_TARGET_MT=1+ flag is used to link some architecture dependant files during the build system configuration (Makefiles creation).

Once your build system is ready, you simply have to run the \verb+make+ command inside the "build" directory (or inside the source directory if you did create the build system inside it) :

\begin{verbatim}
$> make
\end{verbatim}

This will compile both static and dynamic library. This will also compile some examples in the "examples" directory.
Optionally you can install the library on your system using :

\begin{verbatim}
$> sudo make install
[ 38%] Built target xkaapi
[ 76%] Built target xkaapi-static
[ 84%] Built target xkaapi-cpp
[ 92%] Built target xkaapi-cpp-static
[ 93%] Built target branchbound_apiatha
[ 93%] Built target fibo
[ 94%] Built target fibo_adapt
[ 95%] Built target fibo_apiatha
[ 96%] Built target fiboseq
[ 97%] Built target marc_problem
[ 98%] Built target nqueens_apiatha
[100%] Built target transform
Install the project...
-- Install configuration: ""
-- Installing: /usr/local/lib/libxkaapi.so.0.0.1
-- Installing: /usr/local/lib/libxkaapi.so.0
-- Installing: /usr/local/lib/libxkaapi.so
-- Installing: /usr/local/lib/libxkaapi.a
-- Installing: /usr/local/include/kaapi.h
-- Installing: /usr/local/include/kaapi_error.h
-- Installing: /usr/local/include/athapascan-1
-- Installing: /usr/local/include/athapascan-1.h
\end{verbatim}

\subsection{Supported Architectures}

\kaapi currently supports SMP and NUMA architectures. It also compile well on Mac machines (32bits works fine, 64bits is in progress).

\section{Utilization}

In the "examples" directory, there is some examples that are compiled with the library. One may run those examples to see how things works.
Note that not every example in the "examples" directory does work currently.

\subsection{Simple Example}

Let's have a look at the \verb+transform+ example. This code is simply the parallel version (using \kaapi  of course) of the stl::transform
algorithm. Inputs are a table and a unary operation, the unary operation is applied to each single element of the table and the result is stored into an output table.

The library header has to be include (\verb+kaapi.h+). The work is done in the \verb+doit()+ function, in the while loop. In this loop, there's a call to \verb+kaapi_stealpoint()+. If some steal requests have been recieved since the last call to this function, then the \verb+splitter()+ function is called in order to reply to these requests. (...)

The \verb+splitter()+ function will then reply to every requests recieved by browsing the \verb+requests[]+ array.


\subsection{Environnement Variables}

In order to control thread placement and global machine usage, there are two environnement variables that you may use to tune an execution :

\begin{itemize}
\item \verb+KAAPI_CPUCOUNT+ is simply the number of CPU to be used :

\begin{verbatim}
 KAAPI_CPUCOUNT=3 ./transform 100000
\end{verbatim}

\item \verb+KAAPI_CPUSET+ is the set of CPU to be used explicitly if you only want to use a subset of a machine's cores. The value is a comma separated list. You can also specifie a range using "x:y".

\begin{verbatim}
 #this will uses cores 3, 4, 5, 6 and 9 only :
 KAAPI_CPUSET=3,4,5,6,9 ./transform 100000
 
 #this will do the same using range :
 KAAPI_CPUSET=3:6,9 ./transform 100000
\end{verbatim}

\end{itemize}

\subsection{Now}
One can mesure time elapsed for some operations using \verb+kaapi_get_elapsedtime()+ :

\begin{verbatim}
  double t0, t1;
  
  t0 = kaapi_get_elapsedtime();
  //some work to do ...
  t1 = kaapi_get_elapsedtime();

  std::cout << "Time elapsed:" << t1-t0 << std::endl;
\end{verbatim}


\end{document}
