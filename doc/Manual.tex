\documentclass{article}
\usepackage[utf8x]{inputenc}

\usepackage{url} \urlstyle{sf}
\usepackage[a4paper,margin=1.9cm]{geometry}
\usepackage{xspace}
\usepackage[american]{babel}
\usepackage{palatino}
\usepackage{bibtopic}
%\usepackage{graphicx}
%\usepackage{enumitem}
%\usepackage{dot2texi}

\newcommand{\kaapi}{\textsc{X}-Kaapi\xspace}
%%%\newcommand{\new}{\hspace*{10ex}\textbf{\textsc{New in \kaapi.}~\\}\xspace}
%\newcommand{\new}{}
%\newcommand{\inote}[1]{\textit{\textbf{  \center \hrule Implementation note\hrule}}\vspace*{1ex}\textit{#1}\vspace{1ex} \hrule
%\vspace*{2ex}}

%%\newcounter{subsubsection}[subsection]
%\renewcommand{\subsubsection}[1]{~\\ \addtocounter{subsubsection}{1} \noindent\textit{
%%\textbf{\thesubsection.
%\thesubsubsection. #1\\}}

%
%\newcounter{subsubsubsection}[subsubsection]
%\newcommand{\subsubsubsection}[1]{~\\ \addtocounter{subsubsubsection}{1} \noindent\textit{\textbf{\thesubsubsection.
%\thesubsubsubsection. #1\\}}}
%\newtheorem{proposition}{Proposition}

%%\renewcommand{\subsubsection}[1]{~\\ \addtocounter{subsubsection}{1} \noindent\textit{\textbf{\thesubsubsection\hspec #1\\}}}

\begin{document}

\title{\kaapi installation and user guide}
\author{Vincent Danjean, Thierry Gautier, Christophe Laferrière, Fabien Le Mentec}
\date{\today}
\maketitle
\tableofcontents
\newpage

%% intro
\section{Introduction}

\subsection{\kaapi runtime}
\kaapi is a runtime for fine grain parallelism on multicore architectures.
It relies on workstealing paradigms.
The core library comes with a full set of complementary progamming interfaces.
This document is a user manual containing installation instructions and runtime
options, as well as a description of the components shipped with the runtime.

\subsection{Supported Platforms}
\kaapi targets essentially SMP and NUMA platforms. The runtime should run
on every system providing:
\begin{itemize}
\item a GNU toolchain (4.3),
\item the pthread library.
\end{itemize}
It has been extensively tested on the following operating systems:
\begin{itemize}
\item GNU-Linux/x86\_64,
\item MacOSX/PowerPC.
\end{itemize}
There is no version for Windows yet.

%% installation
\section{Installation}

There are 2 ways to install \kaapi:
\begin{itemize}
\item using the debian packages,
\item installing from source.
\end{itemize}

%% sub
\subsection{Using the debian packages}
TODO
\subsubsection{Retrieving the debian packages}
TODO

%% sub
\subsection{Installing from sources}

\subsubsection{Retrieving the sources}
There are 2 ways to retrieve the sources:
\begin{itemize}
\item download a release snapshot at the following url:\newline
\url{https://gforge.inria.fr/frs/?group_id=94}.
\item clone the project git repository:\newline
\verb+> git clone git://git.ligforge.imag.fr/git/kaapi/xkaapi.git xkaapi+
\end{itemize}

\subsubsection{Configuration}
The build system uses GNU Autotools.
In case you cloned the project repository, you first have to bootstrap the
configuration process by running the following script:
\begin{verbatim}
$> ./bootstrap
\end{verbatim}
The \textit{configure} file should be present. It is used to create the
\textit{Makefile} accordingly to your system configuration. Command line
options can be used to modify the default behavior. You can have a complete
list of the available options by running:
\begin{verbatim}
$> ./configure --help
\end{verbatim}
Below is a list of the most important ones:
\begin{itemize} %% option list
\item \verb+--enable-target=mt+\newline
Select the target platform. Defaults to 'mt', for pthread.
\item \verb+--enable-mode=debug+ or \verb+release+\newline
Choose the compilation mode of the library. Defaults to release.
\item \verb+--with-perfcounter+\newline
Enable performance counters support.
\item \verb+--with-papi+\newline
Enable the PAPI library for low level performance counting.
More information on PAPI can be found at http://icl.cs.utk.edu/papi/.
\item \verb+--prefix=+\newline
Overload the default installation path.
\end{itemize} %% option list
Example:
\begin{verbatim}
./configure --enable-mode=release --enable-target=mt --prefix=$HOME/install
\end{verbatim}
If there are errors during the configuration process, you have to solve
them before going further. It is likely there is a missing dependency on your
system, in which case the log gives you the name of the software to install.

\subsubsection{Compilation and installation}
On success, the configuration process generates a Makefile. the 2 following
commands build and install the \kaapi runtime:
\begin{verbatim}
$> make
$> make install
\end{verbatim}

\subsubsection{Checking the installation}
The following checks the runtime is correctly installed on your system:
\begin{verbatim}
$> make check
\end{verbatim}

\subsubsection{Compilation of the examples}
The following compiles the examples applications:
\begin{verbatim}
$> make examples
\end{verbatim}

\section{Using \kaapi}

\subsection{\kaapi integration}
Integrating \kaapi in your project requires 2 steps:
\begin{itemize}
\item include the header files. Example:
\begin{verbatim} 
#include <kaapi.h> /* C version */
#include <kaapi++> // C++ version
\end{verbatim}
\item link with the library. Example:
\begin{verbatim} 
gcc -o main main.c -lxkaapi /* C version */
g++ -o main main.cpp -lxkaapixx // C++ version
\end{verbatim}
\end{itemize}
Refer to the API documentation for information relative
to the \kaapi programming interfaces.

\subsection{Examples}
The directory \textit{examples/} contains sample applications using \kaapi.
Both C and C++ are used. Some direclty lies on top of the core runtime,
while other make use of higher level interfaces. Below is a short
description for some of them:
\begin{itemize}
\item fibo\_xkaapi\_adapt.c\newline
Implementation of the Fibonacci sequence generation using adaptive
task stealing.
\item fibo\_xkaapi.c\newline
Same as above, using DFG tasks.
\item fibo\_kaapixx.cpp, fibo\_kaapixx\_opt.cpp\newline
Same as above, using the \textit{ka} C++ interface.
\item fibo\_atha.cpp\newline
Same as above, using the \textit{Athapascan} interface.
\item nqueens\_apikaapi\newline
Nqueens problem implementation using \textit{ka} C++ interface.
\item nqueens\_apiatha\newline
Same as above, using the \textit{Athapascan} interface.
\item poisson3d-kaapi.cpp\newline
3 dimension Poisson solver implemented using the \textit{ka}
C++ interface.
\item matrix\_multiply\_cilk2kaapi.cpp\newline
Matrix multiplication using loop parallelization. Implemented using
the \textit{ka} C++ interface.
\end{itemize}

\subsection{Runtime environment variables}
The runtime behavior can be driven by using the following
optionnal environment variables.

\begin{itemize} %% env var enumeration

%% KAAPI_CPUCOUNT
\item \verb+KAAPI_CPUCOUNT+\newline
The number of process unit to be used by the runtime.
No assumption is made regarding which unit is used.
Example:
\begin{verbatim}
 KAAPI_CPUCOUNT=3 ./transform 100000
\end{verbatim}

%% KAAPI_CPUSET
\item \verb+KAAPI_CPUSET+\newline
The set of CPU to be used. It consists of a comma
separated list of cpu indices, first index starting
at 0. By default, and if no KAAPI\_CPUCOUNT is given,
all the host cpus are used.
Example:
\begin{verbatim}
 #use cores 3, 4, 5, 6 and 9 only:
 KAAPI_CPUSET=3,4,5,6,9 ./transform 100000
\end{verbatim}

An index range can be used via the ':' token.
Example:
\begin{verbatim}
 #same as above using the range syntax:
 KAAPI_CPUSET=3:6,9 ./transform 100000
\end{verbatim}

You may exclude a cpu from the set by appending the '!' token.
Example:
\begin{verbatim}
 #this will uses cores from 3 to 9, but not 4:
 KAAPI_CPUSET=3:9,!4 ./transform 100000
\end{verbatim}

%% KAAPI_STACKSIZE
\item \verb+KAAPI_STACKSIZE+\newline
Size of the per thread stack, in bytes. By default, this size
is set to 64 kilo bytes.
Example:
\begin{verbatim}
 KAAPI_STACKSIZE=4096 ./transform 100000
\end{verbatim}

%% KAAPI_WSSELECT
\item \verb+KAAPI_WSSELECT+\newline
Name of the victim processor selection algorithm to use.
\begin{itemize}
\item "workload":
Use a user defined workload to driver the vicitm selection algorithm.
\item any other value: falls back to the default random victim selection
algorithm.
\end{itemize}
Example:
\begin{verbatim}
 KAAPI_WSSELECT=workload ./transform 100000
\end{verbatim}
\end{itemize} %% env var enumeration

\subsection{Monitoring performances}
If configured by \url{--with-perfcounter}, the \kaapi library allows to
output performance counters.

\begin{itemize}
\item \verb+KAAPI_DISPLAY_PERF+\newline
If defined, then performance counters are displayed at the end of the execution.
Example:
\begin{verbatim}
 KAAPI_DISPLAY_PERF=1 ./fibo 30
\end{verbatim}

\item \verb+KAAPI_PERF_PAPIES+\newline
Asuming \kaapi was configured using --with-papi, this variable contains a
comma separated list of the PAPI performance counters to use.
Both counter symbolic names and numeric hexadecimal constants can be used.
More information can be found on the PAPI website
(http://icl.cs.utk.edu/papi/).
Note that counter list cannot exceed 3 elements.
Example:
\begin{verbatim}
 KAAPI_PERF_PAPIES=PAPI_TOT_INS,0x80002230,PAPI_L1_DCM ./fibo 30
\end{verbatim}

\end{itemize}


\end{document}
