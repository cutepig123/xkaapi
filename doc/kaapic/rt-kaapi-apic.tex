\documentclass[a4paper, 11pt]{article}

\usepackage{graphicx}
\usepackage{graphics}
\usepackage{verbatim}
\usepackage{listings}
\usepackage{color}
\usepackage[utf8]{inputenc}
\usepackage{RR}
\usepackage{hyperref}

%%\usepackage[frenchb]{babel} % optionnel
%%
%% date de publication du rapport
\RRdate{November 2011}
%%
%% Cas d'une version deux
%% \RRversion{2}
%% date de publication de la version 2
%% \RRdater{Novembre  2006}
\usepackage{listings}

\usepackage{amssymb}
\usepackage{xspace} 
\usepackage{array} 
\usepackage{multirow}
\newcommand\hyph{\nobreak\hskip0pt-\nobreak\hskip0pt\relax}
\newlength\savedwidth
\newcommand\whline{\noalign{\global\savedwidth
  \arrayrulewidth\global\arrayrulewidth 1.5pt}
  \hline \noalign{\global\arrayrulewidth
  \savedwidth}
}
\newcolumntype{I}{!{\vrule width 1.5pt}}
\renewcommand{\arraystretch}{1.5}

\newcommand{\kaapi}{\textsc{X-Kaapi}\xspace}

%%% For all listing figures
\definecolor{MyDarkBlue}{rgb}{0.254901960784314, 0.411764705882353, 0.882352941176471}

\lstset{commentstyle=\color{blue}}
\lstset{language=C}
\lstset{stringstyle=\ttfamily}
\lstset{ classoffset=1
           }
\lstset{ classoffset=2 
           }
\lstset{ classoffset=3
           }
           
\lstset{classoffset=0, showstringspaces=false}

\RRdate{November 2011}

%%
\RRauthor{
Fabien Le Mentec
  \and
Vincent Danjean
  \and
Thierry Gautier
}

\authorhead{Gautier \& Danjean \& Le Mentec}
\RRtitle{\kaapi C programming interface}
\RRetitle{\kaapi C programming interface}
\titlehead{\kaapi C programming interface}

\RRabstract{This report defines the \kaapi C programming interface.
}
\RRresume{The rapport décrit l'interface de programmation C pour \kaapi
}

\RRmotcle{calcul parallel, \kaapi, C}
\RRkeyword{parallel computing, \kaapi, C}
\RRprojets{MOAIS}
\RRdomaine{3} % cas du domaine numero 1
\RRthemeProj{moais} % theme du projet Apics
\RCGrenoble % Grenoble - Rh\^one-Alpes

\begin{document}

% \makeRR % cas d'un rapport de recherche
\makeRT % cas d'un rapport technique.

\tableofcontents
\addtocontents{toc}{\protect\setcounter{tocdepth}{1}}

\newpage
\section{Software installation}\label{sec:userinstall}

\kaapi is both a programming model and a runtime for high performance parallelism targeting multicore and distributed architectures. 
It relies on the work stealing paradigm.
\kaapi was developed in the MOAIS INRIA project by Thierry Gautier, Fabien Le Mentec, Vincent Danjean and Christophe Laferrière in the early stage of the library.

In this report, only the programming model based on the C API is presented.
The runtime library comes also with a full set of complementary programming interfaces: C, C++ and STL-like interfaces. The C++ and STL interfaces, at a higher level than the C interface, may be directly used for developing parallel programs or libraries.

\subsubsection*{Supported Platforms}
\kaapi targets essentially SMP and NUMA platforms. The runtime should run
on any system providing:
\begin{itemize}
\item a GNU toolchain (4.3),
\item the pthread library,
\item Unix based environment.
\end{itemize}
It has been extensively tested on the following operating systems:
\begin{itemize}
\item GNU-Linux with x86\_64 architectures,
\item MacOSX/Intel processor.
\end{itemize}

There is no version for Windows yet.

\subsubsection*{\kaapi Contacts}
If you wish to contact the XKaapi team, please visite the web site at:
\begin{center}
\url{http://kaapi.gforge.inria.fr}
\end{center}



\newpage
\section{Initialization and termination}

\subsection{Synopsis}
\begin{small}
\lstset{commentstyle=\color{blue}}
\lstset{language=C}
\begin{lstlisting}[frame=tb]
int kaapic_init(int flags)
int kaapic_finalize(void)
\end{lstlisting}
\end{small}

\subsection{Description}
\paragraph{}
\textit{kaapic\_init} initializes the runtime. 
It must be called  by the program before using any of the other routines. 
If successful, there must be a
corresponding \textit{kaapic\_finalize} at the end of the program.

\subsection{Parameters}
\begin{itemize}
\item \textit{flags}: if not zero, start only the main thread to avoid disturbing
the execution until tasks are actually scheduled. The other threads are suspended
waiting for a parallel region to be entered (refer to kaapic\_begin\_parallel).
\end{itemize}

\subsection{Return value}
\begin{itemize}
\item [0] in case of success
\item [else] an error code
\end{itemize}

\subsection{Example}
\paragraph{}
\begin{small}
\begin{lstlisting}[frame=tb]
  int main()
  {
    int err = kaapic_init(1);

    ...

    kaapic_finalize();
}
\end{lstlisting}
\end{small}


\newpage
\section{Concurrency}

\subsection{Synopsis}
\begin{small}
\lstset{commentstyle=\color{blue}}
\lstset{language=C}
\begin{lstlisting}[frame=tb]
int  kaapic_get_concurrency(void)
int kaapic_get_thread_num(void)
\end{lstlisting}
\end{small}

\subsection{Description}
\paragraph{}
Concurrency related routines.

\subsection{Return value}
\paragraph{}
\textit{kaapic\_get\_concurrency} returns the number of parallel thread available
to the \kaapi runtime.
\paragraph{}
\textit{kaapic\_get\_thread\_num} returns the current thread identifier. Note it
should only be called in the context of a \kaapi thread.

\subsection{Example}
\paragraph{}
\begin{small}
\begin{lstlisting}[frame=tb]
  int main()
  {
    int err = kaapic_init(1);

   printf("#available threads: %i\n", 
      kaapic_get_concurrency() );
   printf("My thread identifier is: %i\n", 
      kaapic_get_thread_num() );
    ...

    kaapic_finalize();
  }
\end{lstlisting}
\end{small}


\newpage
\section{Performance}

\subsection{Synopsis}
\begin{small}
\lstset{commentstyle=\color{blue}}
\lstset{language=C}
\begin{lstlisting}[frame=tb]
double kaapic_get_time(void)
\end{lstlisting}
\end{small}

\subsection{Description}
\paragraph{}
Capture the current time. Used to measure the time spent in a code region.

\subsection{Parameters}
\paragraph{}
None.

\subsection{Return value}
\paragraph{}
Time in seconds since an arbitrary time in the past.

\subsection{Example}
\paragraph{}
\begin{small}
\begin{lstlisting}[frame=tb]
  int main()
  {
    double start, stop;
    int err = kaapic_init(1);
    start = kaapi_get_time(); 
    ...
    stop = kaapi_get_time(); 


    printf("Time : %f (us)\n", stop-start );
    kaapic_finalize();
  }
\end{lstlisting}
\end{small}


\newpage
\section{Independent loops}

\subsection{Synopsis}
\begin{small}
\lstset{commentstyle=\color{blue}}
\lstset{language=C}
\begin{lstlisting}[frame=tb]
int kaapic_foreach( 
  int first, 
  int last,
  kaapic_foreach_attr_t* attr,
  int32_t nparam,
  ...
 ) 

int kaapic_foreach_withformat( 
  int first, 
  int last,
  kaapic_foreach_attr_t* attr,
  int32_t nparam,
  ...
 ) 
\end{lstlisting}
\end{small}

\subsection{Description}
\paragraph{}
Those routines run a parallel loop over the range [\textit{first}, \textit{last})\footnote{
This is an \textbf{exclusive} interval in the C interface and an \textbf{inclusive} interval in the Fortran interface.}
The loop is given as function with its parameters. The body function has \textit{nparam} parameters and it is 
passed in the \textit{...} optional effective parameter list of the foreach interface.

At runtime, the initial interval is dynamically split in $K$ disjoint intervals $[b_i, e_i)$ such that $\cup_{i=0..K-1} [b_i, e_i) = [first, last)$. For each of this sub-interval, \kaapi calls by several threads \textit{body( $b_i$, $e_i$, tid, $e_0$, ..., $e_{nparam-1}$)} where \textit{tid} is the thread identifier that made the call.

If not null, \textit{attr} is a pointer to an attribut that can be pass tuning parameter to the runtime. Please see \textit{kaapic\_foreach\_attr\_set\_grains}.
\subsection{Parameters}
\paragraph{}
For \textit{kaapi\_foreach} interface, the format of the optional parameter list is:
\begin{description}
\item [body]: the function with signature\\
\hspace*{5ex}\textit{void (*)(int, int, int [,type$_0$, .., type$_{nparam-1}$])}.\\
Each type \textit{type$_i$} could be:
\begin{itemize}
\item a pointer to a memory data
\item a scalar value with size not bigger that the size of a pointer.
\end{itemize}

\item [$e_0$]: first effective parameter passed to \textit{body}.
\item [...]
\item [$e_{nparam-1}$]: last effective parameters passed to \textit{body}.
\end{description}

\paragraph{}
For \textit{kaapi\_foreach\_with\_format} interface extend \textit{kaapi\_foreach} interface in order to pass the size, the type and the access mode of each of the effective parameter. The format of the optional parameter list is:
\begin{description}
\item [body]: the function with signature\\
\hspace*{5ex}\textit{void (*)(int, int, int [,type$_0$, .., type$_{nparam-1}$])}.\\
Each type \textit{type$_i$} could be:
\begin{itemize}
\item a pointer to a memory data
\item a scalar value with size not bigger that the size of a pointer.
\end{itemize}

\item [mode, $e_0$, count, type]: access mode, first effective parameter passed to \textit{body}, the number of type elements pointed by $e_0$ and the type of each element.
\item [...]
\item [mode, $e_{nparam-1}$, count, type]: access mode, last effective parameter passed to \textit{body}, the number of type elements pointed by $e_{nparam-1}$ and the type of each element.
\end{description}
Please refer to the data flow programming section to have a description of \textit{mode} and \textit{type} informations.

\subsection{Return value}
\paragraph{}

In case of success the function return 0, else it returns an error code.

\subsection{Example}
Refer to examples/kaapic\\

\begin{small}
\begin{lstlisting}[frame=tb]
  /* loop body */
  static void body(
    int i, int j, int tid, double* array, double* value
  ) 
  {
    int k;
    for (k = i; k < j; ++k)
      array[k] += *value;
  }

  int main()
  {
    double* array;
    double value;
    int err = kaapic_init(1);
    start = kaapi_get_time(); 
    /* apply body on array[0..size-1] */
    kaapic_foreach( 0, size, 2, body, array, &value );
    stop = kaapi_get_time(); 

    printf("Time : %f (us)\n", stop-start );
    kaapic_finalize();
  }
\end{lstlisting}
\end{small}

The next example is equivalent to the previous:
\begin{small}
\begin{lstlisting}[frame=tb]
  /* loop body */
  static void body(
    int i, int j, int tid, double* array, double* value
  ) 
  {
    int k;
    for (k = i; k < j; ++k)
      array[k] += *value;
  }

  int main()
  {
    double* array;
    double value;
    int err = kaapic_init(1);
    start = kaapi_get_time(); 
    /* apply body on array[0..size-1] */
    kaapic_foreach_with_format( 0, size, 2, body, 
        KAAPIC_MODE_RW, array, size, KAAPIC_TYPE_DOUBLE
        KAAPIC_MODE_V, &value, 1,  KAAPIC_TYPE_DOUBLE
    );
    stop = kaapi_get_time(); 

    printf("Time : %f (us)\n", stop-start );
    kaapic_finalize();
  }
\end{lstlisting}
\end{small}



\newpage
\section{Dataflow programming}
\subsection{Synopsis}
\begin{small}
\lstset{commentstyle=\color{blue}}
\lstset{language=C}
\begin{lstlisting}[frame=tb]
int kaapic_spawn(int32_t nargs, ... )
\end{lstlisting}
\end{small}

\subsection{Description}
\paragraph{}
Create a new computation task implemented by a call to a function \textit{body} with effective parameters \textit{e$_i$}.

The function \textit{body} as well as its effective parameters are pass in the optional parameter list of \textit{kaapic\_spawn}.
The format of the optional parameter list is:
\begin{description}
\item [body]: the function with signature\\
\hspace*{5ex}\textit{void (*)([type$_0$, .., type$_{nparam-1}$])}.\\
Each type \textit{type$_i$} could be:
\begin{itemize}
\item a pointer to a memory data
\item a scalar value with size not bigger that the size of a pointer.
\end{itemize}
\end{description}

\textit{body} is called with the user specified arguments, there is no argument added by \kaapi:\\
\begin{small}
\lstset{language=C}
\begin{lstlisting}[frame=tb]
body(e0, e1, ..., )
\end{lstlisting}
\end{small}

\subsubsection{Format of each 4 successive arguments}
\paragraph{}
Each task parameter is described by 4 successive arguments including:
\begin{itemize}
\item the access \textit{mode}.
\item the argument \textit{value},
\item the element \textit{cound},
\item the parameter \textit{type}
\end{itemize}

\subsubsection{Mode information}
\paragraph{}
The parameter \textit{mode} is one of the following:
\begin{itemize}
\item KAAPIC\_MODE\_R=0 for a read access,
\item KAAPIC\_MODE\_W=1 for a write access,
\item KAAPIC\_MODE\_RW=2 for a read write access,
\item KAAPIC\_MODE\_V=3 for a parameter passed by value.
\end{itemize}

\subsubsection{Type information}
\paragraph{}
The \textit{type} is one of the following:
\begin{itemize}
\item KAAPIC\_TYPE\_CHAR=0,
\item KAAPIC\_TYPE\_INT=1,
\item KAAPIC\_TYPE\_REAL=2,
\item KAAPIC\_TYPE\_DOUBLE=3.
\end{itemize}

\paragraph{}
If a parameter is an array, \textit{count} must be set to the number of the element of the array.
For a scalar value, it must be set to 1.


\subsection{Parameters}
\begin{itemize}
\item \textit{nargs}: the argument count.
\item \textit{...}: the \textit{body} followed by a \textit{mode}, \textit{value}, \textit{count} , \textit{type}  tuple list.
\end{itemize}

\subsection{Return value}
\paragraph{}
None.

\subsection{Example}\label{fibo}
\paragraph{}
Refer to examples/kaapif/dfg

\begin{small}
\begin{lstlisting}[frame=tb]
  /* computation task entry point */
  void fibonacci(int n, int* result)
  {
    /* task user specific code */
    if (n <2)
      *result = n;
     else 
     {
        int result1, result2;
        kaapic_spawn( 2, fibonacci, 
          KAAPIC_MODE_V, n-1, 1, KAAPIC_TYPE_INT
          KAAPIC_MODE_W, &result1, 1,  KAAPIC_TYPE_INT
        );
        kaapic_spawn( 2, fibonacci, 
          KAAPIC_MODE_V, n-2, 1, KAAPIC_TYPE_INT
          KAAPIC_MODE_W, &result2, 1,  KAAPIC_TYPE_INT
        );
        kaapic_sync();
        *result = result1 + result2;
     }
  }

  int main()
  {
    int n = 30;
    int result= 0;
    int err = kaapic_init(1);

    start = kaapi_get_time(); 
    /* apply body on array[0..size-1] */
    kaapic_spawn( 2, fibonacci, 
        KAAPIC_MODE_V, n, 1, KAAPIC_TYPE_INT
        KAAPIC_MODE_W, &result, 1,  KAAPIC_TYPE_INT
    );
    stop = kaapi_get_time(); 

    printf("Time : %f (us)\n", stop-start );
    kaapic_finalize();
  }
\end{lstlisting}
\end{small}


\newpage
\section{Parallel regions}

\subsection{Synopsis}
\begin{small}
\lstset{commentstyle=\color{blue}}
\lstset{language=C}
\begin{lstlisting}[frame=tb]
int kaapic_begin_parallel(void)
int kaapic_end_parallel( int flag )
\end{lstlisting}
\end{small}

\subsection{Description}
\paragraph{}
\textit{kaapic\_begin\_parallel} and \textit{kaapic\_end\_parallel} mark the
start and the end of a parallel region. Regions are used to wakeup and suspend
the \kaapi system threads so they avoid disturbing the application when idle.
This is important if another parallel library is being used. Wether threads
are suspendable or not is controlled according by the \textit{kaapi\_init} parameter.

\subsection{Parameters}
\begin{itemize}
\item \textit{falg}: if zero, an implicit synchronization is inserted before leaving the region.
\end{itemize}

\subsection{Return value}
\paragraph{}
None.

\subsection{Example}
\paragraph{}
\begin{small}
\begin{lstlisting}[frame=tb]
  int main()
  {
    int err = kaapic_init(1);

    kaapic_begin_parallel();
    ...
    kaapic_end_parallel();
    ...

  }
\end{lstlisting}
\end{small}


\newpage
\section{Synchronization}

\subsection{Synopsis}
\begin{small}
\lstset{commentstyle=\color{blue}}
\lstset{language=C}
\begin{lstlisting}[frame=tb]
void kaapic_sync(void)
\end{lstlisting}
\end{small}

\subsection{Description}
\paragraph{}
Synchronize the sequential with the parallel execution flow. When this routine
returns, every computation task has been executed and memory is consistent for
the processor executing the sequential flow.

\subsection{Return value}
\paragraph{}
None.

\subsection{Example}
\paragraph{}
Refer to the Fibonacci example in section~\ref{fibo}.


\end{document}
